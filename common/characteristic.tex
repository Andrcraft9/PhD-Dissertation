
{\actuality}
К важнейшим проблемам XXI века относится решение задачи прогноза изменения климата, в значительной мере обусловленное антропогенным воздействием,
связанным с выбросами в атмосферу парниковых газов и других загрязняющих веществ, это можно наглядно увидеть в многочисленных отчетах
IPCC (Intergovernmental Panel on Climate Change) \optcite{IPCC21}.
Одним из основных современных инструментариев исследования изменчивости климата, понимания его прошлых изменений и прогнозирования будущих являются модели земной системы (МЗС),
главными компонентами которой являются модели общей циркуляции атмосферы и океана.
При этом в силу своих пространственных масштабов и физических свойств атмосфера служит «генератором» изменений климата,
а океан – основным «накопителем» этих изменений. Кроме того, океан в своем взаимодействии с атмосферой отвечает за генерацию десятилетних
и мультидесятилетних колебаний климата.
Поэтому создание вычислительно эффективной модели общей циркуляции океана (МОЦО), воспроизводящей сложную гидротермодинамику
Мирового океана является крайне важной задачей. Кроме того, создание эффективной модели морской гидротермодинамики на основе современных достижений
численного моделирования важно и для изучения физических процессов, формирующих региональную циркуляцию морей и океанов, что,
в свою очередь, необходимо для потребностей судоходства, рыболовства, оперативной океанографии и другие.

Современное интенсивное развитие климатических моделей и, в частности МОЦО, в настоящее время связано в первую очередь с бурным развитием вычислительной техники.
Появление терафлопных и петафлопных вычислительных систем, т.н. суперкомпьютеров, открыло возможности для построения глобальных моделей с
высоким пространственным разрешением, которые позволяют описывать мезо- и субмезо масштабы вихревой изменчивости океана и проводить с ними расчеты на долгие сроки.
На сегодняшний день большинство высокопроизводительных вычислительных систем являются гетерогенными, объединяющими различные типы вычислительных процессоров.
Это можно наглядно увидеть из списка TOP 500 самых мощных суперкомпьютеров в мире \optcite{TOP500}.
Такие системы в общем случае могут состоять из большого количества процессоров разного типа.
В наши дни выделяют основное направление развития гетерогенных систем, состоящее в совместном использовании многоядерного центрального процессора (CPU)
и массивно-параллельных ускорителей, например графических процессоров (GPU).
Суперкомпьютерная техника стремительно развивается в России и тенденция развития схожа с мировой - это можно видеть из списка ТОП 50
самых мощных суперкомпьютеров в СНГ \optcite{TOP50}.
Мощнейший суперкомпьютер России «Ломоносов-2» в МГУ им. М.В.Ломоносова являются вычислительной системой именно гетерогенного типа \optcite{L2}.
Создание модели гидродинамики океана, использующую эффективно ресурсы таких гетерогенных вычислительных систем,
является сложной и актуальной на сегодняшний день задачей.

Основной целью настоящего проекта является усовершенствование российской сигма-модели общей циркуляции океана INMOM
(Institute of Numerical Mathematics Ocean Model) для эффективного использования на массивно-параллельных и гетерогенных вычислительных системах \optcite{INMOM}.
Модель INMOM разрабатывается в ИВМ РАН (Институт вычислительной математики РАН) и уже на протяжении более полутора десятков лет используется
в качестве океанического блока МЗС INMCM (Institute of Numerical Mathematic Climate Model) различных версий \optcite{VolodinINMCM2013}.
Именно эта совместная модель является пока единственным представителем от России в различных этапах международного проекта сравнения климатических моделей CMIP (Coupled Model Intercomparison Project), проводящегося под эгидой IPCC (International Panel on Climate Change, или в русской транскрипции МГЭИК – Межправительственная группа экспертов по изменению климата).
Последнее такое сравнение CMIP6 (CMIP, Phase 6) включено в шестой отчет IPCC, недавно проходившего в 2010-2013 гг. под эгидой IPCC (International Panel on Climate Change,
или в русской транскрипции МГЭИК – Межправительственная группа экспертов по изменению климата).
Модель написана на языке Fortran 90/95.

Следует особо подчеркнуть оригинальность как сигма-модели общей циркуляции океана INMOM, так и самой модели климата INMCM, разработанных в ИВМ РАН.
Возникающий при этом в рамках CMIP «параллелизм» необходим для контроля воспроизводимости получаемых результатов и для статистического исключения возможных
ошибок прогноза изменений климата. Это особо ценится в проекте CMIP, т.к. позволяет учитывать больший спектр климатической изменчивости.
Именно для этого проводилось и проводится сравнение результатов моделирования климата и его изменений с помощью различных МЗС в рамках международных программ,
являющиеся клубами высоких технологий. С этой точки зрения следует особо подчеркнуть,
что INMOM - это единственная сигма–координатная модель, способная адекватно воспроизводить климатическую циркуляцию Мирового океана при расчетах на большие времена.

%Модель относится к классу сигма-моделей океана:
%в ней в качестве вертикальной переменной используется безразмерная величина $\sigma \in [0, 1]$, которая определяется из соотношения:
%$$ \sigma = \frac{z + \zeta}{H + \zeta} $$
%где $z$ - направленная вниз обычная вертикальная координата по глубине, с началом на невозмущенной поверхности океана;
%$\zeta$ - отклонение уровня океана от невозмущенной поверхности; 
%$H$ - глубина океана в состоянии покоя.
%Модель написана на языке Fortran 90/95.
%Предыдущая версия модели INMOM используется в качестве океанического блока 
%климатической модели INMCM (Institute of Numerical Mathematics Climate Model), созданной в ИВМ РАН и участвующей в программе
%IPCC по прогнозированию изменений климата \optcite{VolodinINMCM2013}.

В работе отдельно рассматривается система нелинейных уравнений мелкой воды,
являющаяся блоком сигма-модели общей циркуляции океана INMОM.
Система уравнений мелкой воды является неотъемлемой и одной из самых важных подзадач в моделях общей циркуляции океана \optcite{ROMS2005}, \optcite{Ibraev2015}.
Решение этой системы уравнений занимает существенную часть времени, необходимого для решения полной задачи циркуляции океана,
как было показано, например, в работе \cite{ChaplyginINMOM2017}.
Кроме того, на основе нелинейных уравнений мелкой воды реализуются наиболее продвинутые системы
предупреждения о цунами \optcite{TUNAMI}, реализованные для Мирового океана с высоким пространственным разрешением.
Поэтому возникает вопрос об эффективной параллельной реализации алгоритма решения системы уравнений мелкой воды.

Следует особо подчеркнуть необходимость развития суперкомпьютерных технологий в России и развития отечественных моделей, поскольку это является необходимым условием обеспечения независимой экспертизы формирования климатических изменений и изменчивости циркуляции Мирового океана как в глобальном, так и на региональном масштабах.
%что, в свою очередь, является необходимым условием национальной безопасности России.

% {\progress}
% Этот раздел должен быть отдельным структурным элементом по
% ГОСТ, но он, как правило, включается в описание актуальности
% темы. Нужен он отдельным структурынм элемементом или нет ---
% смотрите другие диссертации вашего совета, скорее всего не нужен.

{\aim} данной работы является разработка усовершенствованной версии сигма-модели общей циркуляции океана INMOM и модели мелкой воды для эффективного использования на массивно-параллельных многопроцессорных и гетерогенных вычислительных системах.

%Целью настоящего проекта является исследовать эффективность работы усовершенствованной версии INMOM на разных типах вычислительных систем, провести анализ масштабируемости и производительности. Провести численные эксперименты с этой моделью по воспроизведению циркуляции океана с вихредопускающим и вихреразрешающим пространственным разрешением около 25 и 10 км соответственно.
%Подготовить усовершенствованную версию INMOM к включению в качестве нового океанического блока в МЗС INMCM.
%Планируемые эксперименты с моделью INMOM позволят также продвинутся в ее дальнейшем улучшении.
%Следует особо подчеркнуть необходимость развития суперкомпьютерных технологий в России и развития отечественных моделей,
%поскольку это является необходимым условием обеспечения независимой экспертизы формирования климатических изменений и изменчивости циркуляции
%Мирового океана как в глобальном, так и на региональном масштабах, что, в свою очередь, является необходимым условием национальной безопасности России.

Для~достижения поставленной цели необходимо было решить следующие {\tasks}:
\begin{enumerate}[beginpenalty=10000] % https://tex.stackexchange.com/a/476052/104425
    \item Разработать программную архитектуру сигма-модели общей циркуляции океана INMOM и, в частности, модели мелкой воды, предполагающую гибкий переход на гибридные модели параллельного программирования с использованием технологий MPI, OpenMP, CUDA.
    \item Разработать параллельные вычислительные алгоритмы решения нелинейных уравнений мелкой воды для использования на массивно-параллельных многопроцессорных и гетерогенных вычислительных системах.
    \item Разработать метод балансировки нагрузки вычислений на процессорах для улучшения масштабируемости и производительности моделей на высокопроизводительных вычислительных системах.
    \item На основе разработанных параллельных вычислительных алгоритмов решения нелинейных уравнений мелкой воды разработать усовершенствованную версию сигма-модели общей циркуляции океана INMOM для эффективного использования на массивно-параллельных многопроцессорных вычислительных системах.
    \item Исследовать масштабируемость и производительность предложенных методов на массивно-параллельных и гетерогенных вычислительных системах. 
%\item Разработать гибридные модели параллельного программирования для эффективного использования на массивно-параллельных многопроцессорных  и гетерогенных вычислительных системах.
\end{enumerate}

%1. Создание эффективной параллельной реализации уравнений мелкой воды, которые являются одним из основных блоков INMOM. Система уравнений мелкой воды является неотъемлемой и одной из самых важных подзадач не только в INMOM, но и в любой другой МОЦО. Решение этой системы уравнений занимает существенную часть времени, необходимого для решения полной задачи циркуляции океана. Кроме того, на основе нелинейных уравнений мелкой воды реализуются наиболее продвинутые системы предупреждения о цунами, реализованные для Мирового океана с высоким пространственным разрешением. Поэтому вопрос об эффективной параллельной реализации алгоритма решения системы уравнений мелкой воды особенно актуален.
%2. Создание эффективной параллельной реализации всех блоков INMOM. При построении усовершенствованной модели циркуляции океана за основу будут взяты параллельные методы и подходы, применяемые в модели мелкой воды из предыдущей задачи. Обобщение таких методов и подходов от двумерного случая мелкой воды к трехмерной задаче циркуляции океана будет происходить без существенных изменений и сложностей, т.к. модель океана INMOM является сигма-моделью с одинаковым числом расчетных уровней по глубине для всех точек сетки по горизонтали.
%3. Тестирование усовершенствованной модели циркуляции океана INMOM на различных массивно-параллельных и гетерогенных вычислительных системах. В качестве основных платформ для тестирования планируется использовать суперкомпьютеры «Ломоносов» и «Ломоносов-2», а также суперкомпьютеры ИВМ РАН и МСЦ РАН. Будет проведен комплексный анализ масштабируемости и производительности усовершенствованной модели океана на различных вычислительных системах.
%4. Проведение численных экспериментов по воспроизведению циркуляции океана с использованием усовершенствованной модели INMOM. Для этого будет реализована версии INMOM для акватории Мирового океана с вихредопускающим пространственным разрешением, а для Атлантического океана и с вихреразрешающим разрешением. Будет проведен комплексный анализ полученных результатов, которые позволят продвинуться как в дальнейшем улучшении эффективности модели, так и в решении ряда практических и научных задач.
%5. Подготовка усовершенствованной версии INMOM к включению в качестве нового океанического блока в МЗС INMCM. Для каждой совместной модели задача интерполяции данных между сетками является крайне важной. Поэтому в рамках этой задачи будет предложен и реализован эффективный параллельный алгоритм интерполяции для произвольных пар прямоугольных сеток.

{\novelty}
\begin{enumerate}[beginpenalty=10000] % https://tex.stackexchange.com/a/476052/104425
    \item Разработаны параллельные алгоритмы решения нелинейных уравнений мелкой воды для эффективного использования на массивно-параллельных и гетерогенных вычислительных системах, представленные в виде отдельного программного комплекса. Этот программный комплекс можно использовать как в качестве блока сигма-модели циркуляции океана INMOM, так и независимо.
    \item Разработана усовершенствованная версия российской сигмы-модели общей циркуляции океана INMOM для эффективного использования на массивно-параллельных многопроцессорных вычислительных системах.
    \item Разработан метод балансировки нагрузки вычислений на процессорах для улучшения масштабируемости и производительности сигмы-модели общей циркуляции океана INMOM на высокопроизводительных вычислительных системах.
\end{enumerate}

{\influence}
Разработанный программный комплекс решения нелинейных уравнений воды можно использовать как в качестве блока сигма-модели циркуляции океана INMOM, так и независимо, например для расчетов прохождения волны цунами, штормов, приливов и ветрового нагона. С помощью разработанного программного комплекса проводилось моделирование цунами в Японии и экстремального шторма на Азовском море, было показано, что результаты вычислений согласуются с данными наблюдений.
Разработанная усовершенствованная сигма-модель общей циркуляции океана INMOM может эффективно использоваться на массивно-параллельных многопроцессорных вычислительных системах, что позволяет существенно сократить время расчетов без ухудшения точности результатов. Эффективность параллельной реализации подтверждена тестами на современных российских суперкомьютерах. На основе усовершенствованной сигма-модели океана INMOM была построена система оперативного моделирования Северного Ледовитого океана и прилегающих к нему акваторий (INMOM-Арктика). Результаты сравнения расчетов и данных наблюдений свидетельствуют, что модель INMOM-Арктика позволяет корректно воспроизводить циркуляцию Северного Ледовитого океана.
Предполагается, что усовершенствованная версия сигма-модели океана INMOM будет использована в качестве нового океанического блока в МЗС INMCM.

{\methods}
Теория и методы вычислительной математики; численные эксперименты на современных суперкомпьютерах; современные инструменты для разработки, отладки и профилирования программного комплекса на распределенных и гетерогенных вычислительных системах.

{\defpositions}
\begin{enumerate}[beginpenalty=10000] % https://tex.stackexchange.com/a/476052/104425
    \item Разработана программная архитектура сигма-модели общей циркуляции океана INMOM и, в частности, модели мелкой воды, предполагающая гибкий переход на гибридные модели параллельного программирования с использованием технологий MPI, OpenMP, CUDA.    
    \item Разработаны параллельные вычислительные алгоритмы решения нелинейных уравнений мелкой воды для использования на массивно-параллельных многопроцессорных и гетерогенных вычислительных системах.
    \item Разработан метод балансировки нагрузки вычислений на процессорах для улучшения масштабируемости и производительности моделей на высокопроизводительных вычислительных системах.
    \item На основе разработанных параллельных вычислительных алгоритмов решения нелинейных уравнений мелкой воды разработана усовершенствованная версия сигма-модель общей циркуляции океана INMOM для эффективного использования на массивно-параллельных многопроцессорных вычислительных системах.
    \item Исследована масштабируемость и производительность предложенных методов на массивно-параллельных и гетерогенных вычислительных системах.     
%\item Разработаны гибридные модели параллельного программирования для эффективного использования на массивно-параллельных многопроцессорных и гетерогенных вычислительных системах.
\end{enumerate}
%В папке Documents можно ознакомиться с решением совета из Томского~ГУ
%(в~файле \verb+Def_positions.pdf+), где обоснованно даются рекомендации
%по~формулировкам защищаемых положений.

{\reliability} полученных результатов обеспечивается использованием в работе теории численных методов, а также сравнением результатов вычислительных экспериментов с данными наблюдений. Материал, изложенный в диссертации, опирается на широкий список научной литературы, посвящённый рассматриваемым методам.

{\probation}
Основные результаты работы докладывались~на:
\begin{enumerate}[beginpenalty=10000]
\item Конференции "Труды 60-й Всероссийской научной конференции МФТИ" (г. Москва, 2017).
\item Конференции "Труды VI международной научно-практической конференции Морские исследования и образование: MARESEDU-2017" (2017).
\item Международной конференции "EGU 2019" (2019).
\item Семинаре "Новые подходы к измерениям и моделированию геофизической турбулентности" (г. Москва, НИВЦ МГУ, 2019).
\item Международной конференции "Математическое моделирование и суперкомпьютерные технологии" (г. Нижний Новгород, 2020).
\end{enumerate}

{\contribution} Автором лично разработаны параллельные алгоритмы решения нелинейных уравнений мелкой воды для эффективного использования на массивно-параллельных и гетерогенных вычислительных системах и разработана усовершенствованная версия российской сигма-модели общей циркуляции океана INMOM для эффективного использования на массивно-параллельных многопроцессорных вычислительных системах. Представленная диссертация является самостоятельным законченным трудом автора. 

{\publications}
Основные результаты по теме диссертации изложены
в~7~печатных изданиях: \cite{ChaplyginINMOM2017}, \cite{ChaplyginSW2017}, \cite{ChaplyginAzov2017},\cite{DianskyInertOsc2019}, \cite{ChaplyginLB2019}, \cite{ChaplyginSW2021}, \cite{Chaplygin_Gusev_Diansky_2022}, 
5 из которых изданы в журналах, рекомендованных ВАК, 2 индексируются в международных базах данных Scopus или Web of Science.
