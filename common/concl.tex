%% Согласно ГОСТ Р 7.0.11-2011:
%% 5.3.3 В заключении диссертации излагают итоги выполненного исследования, рекомендации, перспективы дальнейшей разработки темы.
%% 9.2.3 В заключении автореферата диссертации излагают итоги данного исследования, рекомендации и перспективы дальнейшей разработки темы.
%\begin{enumerate}
%  \item На основе анализа \ldots
%  \item Численные исследования показали, что \ldots
%  \item Математическое моделирование показало \ldots
%  \item Для выполнения поставленных задач был создан \ldots
%\end{enumerate}

%В работе разработана программная архитектура на принципе разделения обязанностей для модели общей циркуляции океана INMOM и, в частности,  модели мелкой воды. Разработанная программная архитектура предоставляет гибкий переход на гибридные модели параллельного программирования с использованием технологий MPI, OpenMP, CUDA.
%На основе программной архитектуры были разработаны гибридные модели параллельного программирования для эффективного использования на массивно-параллельных многопроцессорных и гетерогенных вычислительных системах.
%Был также разработан метод балансировки нагрузки вычислений на процессорах для улучшения масштабируемости и производительности моделей на высокопроизводительных вычислительных системах
%В работе была исследована масштабируемость и производительность предложенных методов на массивно-параллельных и гетерогенных вычислительных системах.

\begin{enumerate}
\item В работе разработана программная архитектура на принципе разделения обязанностей, которая предоставляет гибкий переход на гибридные модели параллельного программирования с использованием технологий MPI, OpenMP, CUDA.
\item На основе программной архитектуры были разработаны параллельные вычислительные алгоритмы решения нелинейных уравнение мелкой воды для эффективного использования на массивно-параллельных многопроцессорных и гетерогенных вычислительных системах:
чистый MPI подход;
гибридный MPI-OpenMP подход для использования на многопроцессорных систем с общей памя­тью;
гибридный синхронный MPI-OpenMP-CUDA подход для использования на гетерогенных вычислительных системах;
гибридный асинхронный MPI-CUDA с перекрытием передачи данных и вычислений;
гибридный MPI-OpenMP-CUDA подход для использования на гетерогенных вычислительных системах с несколькими GPU на вычислительном узле.
\item Исследована масштабируемость и производительность предложенных методов на массивно-параллельных и гетерогенных вычислительных системах. Продемонстрирована масштабируемость модели мелкой воды на CPU и GPU: производительность на одну точку сетки на GPU резко снижается после $2^{19}$ точек на узел, в то время как производительность на CPU отлично масштабируется до $2^{17}$ точек. Тем не менее, вычисления на GPU превосходят вычисления на CPU в 4,7 раза на 30 узлах, использующих 360 ядер CPU и 60 GPU при размере сетки 6100 $\times$ 4460. Также было подремонтировано, что, за счёт перекрытия вычислений с передачей данных, асинхронный подход MPI-OpenMP-CUDA лучше масштабируется до 8 узлов и на 28\% производительнее на 8 графических процессорах, чем синхронный подход без перекрытия вычислений с передачей данных.
\item На основе разработанных параллельных вычислительных алгоритмов решения нелинейных уравнений мелкой воды была разработана усовершенствованная версия сигма-модели общей циркуляции океана INMOM для эффективного использования на массивно-параллельных многопроцессорных вычислительных системах.
\item Разработан метод балансировки нагрузки вычислений на процессорах для улучшения масштабируемости и производительности моделей мелкой воды и общей циркуляции океана INMOM на высокопроизводительных вычислительных системах. Был реализован блочный подход, который имеет ряд преимуществ, например, наиболее эффективная работа с кэш памятью. Показана большая эффективность разработанного метода балансировки нагрузки в сравнении с равномерным разбиением без балансировки нагрузки для сигма-модели океана INMOM: ускорение более чем в 1.7 раза.
\end{enumerate}
