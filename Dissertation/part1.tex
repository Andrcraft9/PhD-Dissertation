\chapter{Модель мелкой воды как составная часть модели гидротермодинамики океана}\label{ch:ch1}

\section{Математическая постановка задачи}\label{sec:ch1/sec1}

Рассматриваемая в работе модель основана на системе нелинейных уравнений мелкой воды, которая записывается в произвольной ортогональной системе координат
в следующем виде:

\begin{equation} \label{eq:ch1/sec1/1}
    \begin{array}{c}
        \displaystyle{ \d{r_x r_y h u}{t} + T_u(u, v) - F_u(u, v) - h r_x r_y l v + r_y h g \d{\zeta}{x} = RHS_u } \\

        \displaystyle{ \d{r_x r_y h v}{t} + T_v(u, v) - F_v(u, v) + h r_x r_y l u + r_x h g \d{\zeta}{y} = RHS_v } \\

        \displaystyle{ \d{h}{t} + \frac{1}{r_x r_y} \left( \d{u r_y h}{x} + \d{v r_x h}{y} \right) = 0 }
    \end{array}
\end{equation}

Где $r_x, r_y$ - метрические коэффициенты Ламе, возникающие при записи системы уравнений в произвольной ортогональной системе координат;
$u$, $v$ – компоненты усредненного по глубине вектора горизонтальной скорости; $l$ – параметр Кориолиса;
$g$ – ускорение свободного падения; $\zeta$ – отклонение уровня моря относительно невозмущённого состояния;
$h = H + \zeta$ – полная глубина океана; $H$ – глубина океана в состоянии покоя.
Операторы переноса $T_u$ , $T_v$ записываются в криволинейной системе координат в дивергентной форме:
\begin{equation} \label{eq:ch1/sec1/2}
    \begin{array}{c}
        \displaystyle{T_{u} (u,v,h)=\frac{\partial hr_{y} uu}{\partial x} +\frac{\partial hr_{x} vu}{\partial y} - h\left(v\frac{\partial r_{y} }{\partial x} -u\frac{\partial r_{x} }{\partial y} \right)v} \\

        \displaystyle{T_{v} (u,v,h)=\frac{\partial hr_{y} uv}{\partial x} +\frac{\partial hr_{x} vv}{\partial y} + h\left(v\frac{\partial r_{y} }{\partial x} -u\frac{\partial r_{x} }{\partial y} \right)u}
    \end{array}
\end{equation}

Операторы вязкости $F_u$ , $F_v$ записываются как дивергенция тензора напряжений:
\begin{equation} \label{eq:ch1/sec1/3}
    \begin{array}{c}
        \displaystyle{F_{u} (u,v)=\frac{1}{r_{y} } \frac{\partial }{\partial x} \left(r_{y} ^{2} KD_{T} h\right)+\frac{1}{r_{x} } \frac{\partial }{\partial y} \left(r_{x} ^{2} KD_{S} h\right)} \\

        \displaystyle{F_{v} (u,v)=-\frac{1}{r_{x} } \frac{\partial }{\partial y} \left(r_{x} ^{2} KD_{T} h\right)+\frac{1}{r_{y} } \frac{\partial }{\partial x} \left(r_{y} ^{2} KD_{S} h\right)}
    \end{array}
\end{equation}

Здесь $\textit{K}$ - коэффициент вязкости, а $D_T$ и $D_S$ - компоненты тензоров напряжений сжатия-растяжения и сдвига соответственно:
\begin{equation} \label{eq:ch1/sec1/4}
    \begin{array}{c}
        \displaystyle{D_{T} =\frac{r_{y} }{r_{x} } \frac{\partial }{\partial x} \left(\frac{u}{r_{y} } \right)-\frac{r_{x} }{r_{y} } \frac{\partial }{\partial y} \left(\frac{v}{r_{x} } \right)} \\

        \displaystyle{D_{S} =\frac{r_{x} }{r_{y} } \frac{\partial }{\partial y} \left(\frac{u}{r_{x} } \right)+\frac{r_{y} }{r_{x} } \frac{\partial }{\partial x} \left(\frac{v}{r_{y} } \right)}
    \end{array}
\end{equation}

В общем случае в правых частях $RHS_u$ , $RHS_v$ рассчитываются градиенты атмосферного
давления и напряжения трения ветра. 
\begin{equation} \label{eq:ch1/sec1/5}
\begin{array}{c} 
\displaystyle{RHS_u = P_{x} + \tau _{x}^{surf}} \\ 

\displaystyle{RHS_v = P_{y} + \tau _{y}^{surf}} 
\end{array} 
\end{equation} 
Здесь $P_x, P_y$ - градиенты атмосферного давления на поверхности океана, $\tau^{surf}$ - напряжение силы ветра на поверхности.
В правых частях также могут рассчитываться и приливные силы,
рассчитываемые через приливной потенциал.

На берегах для скорости задаются граничные условия непротекания и свободного скольжения.

Именно в виде \cref{eq:ch1/sec1/1} - \cref{eq:ch1/sec1/4} нелинейные уравнения мелкой воды представлены в сигма модели общей циркуляции океана INMOM,
возникающие при разрешении быстрых баротропных гравитационных волн \cite{INMOM}, \cite{ChaplyginSW2017}.
Важно отметить, что все переменные в данной постановке задачи двумерные.
В противном случае правые части должны содержать интегралы по глубине от нелинейного взаимодействия трёхмерных величин, в частности, адвективных слагаемых.

\section{Описание численной реализации}\label{sec:ch1/sec2}

\subsection{Дискретизация по пространству}\label{sec:ch1/sec2-1}

Система уравнений \cref{eq:ch1/sec1/1} - \cref{eq:ch1/sec1/4} в модели разрешается с использованием численных методов.
При дискретизации по пространству нелинейных уравнений мелкой воды используется сетка в общем случае нерегулярная по долготе и широте.
Разобьём область $\{ x \in [x_0, x_{max}], \quad y \in [y_0, y_{max}] \}$, куда входит область, на которой рассматривается система уравнений \cref{eq:ch1/sec1/1},
на элементарные ячейки, которые будут иметь форму прямоугольников:

$$ \{ (x, y) : x_{m-1} < x < x_{m}, \quad y_{n-1} < y < y_{n} \} $$

Для решения системы уравнений \cref{eq:ch1/sec1/1} применяется техника построения разностных аппроксимаций по пространству второго порядка точности на разнесенной
'C'-сетке по классификации Аракавы \cite{ARAKAWA1976}, \cite{ARAKAWA1977}. На рис. \cref{fig:grid} показаны распределения переменных в каждой сеточной ячейке.
В центре ячейки задаются $\zeta$ и $H$. На серединах боковых сторон ячейки задаются
компоненты вектора скорости $(u, v)$ и потоковые переменные.
Параметр Кориолиса $l$ определяется в угловых точках.

\begin{figure}[htb!]
    \center
    \includegraphics[scale = 0.6]{CgridBig.png}
    \caption{Распределение переменных на ячейке модельной сетки}
    \label{fig:grid}
\end{figure}

При построении разностных схем особое место уделяется тому, чтобы
в разностных аналогах дифференциальных операторов сохранялись свойства симметрии,
которые выполняются для исходной дифференциальной задачи.
Это позволяет в разностной задаче автоматически удовлетворять энергетическим соотношениям, справедливым для дифференциальной.
Методика построения пространственных разностных аппроксимаций хорошо изложена, например, здесь \cite{ROUCH}, \cite{MARCHUK}.

\subsection{Дискретизация по времени}\label{sec:ch1/sec2-2}

При дискретизации нелинейных уравнений мелкой воды в качестве схемы по времени  используется схема 'Чехарда со средней точкой' ('leapfrog').
Для определения решения на шаге $n+1$ используются решения на шагах $n$ и $n-1$.

Рассмотрим простейшее уравнение адвекции:
\begin{equation} \label{eq:ch1/sec1/6}
    \frac{dU}{dt} = F(U)
\end{equation}
Применяя численную схему по времени 'Чехарда со средней точкой', получим следующее:
\begin{equation} \label{eq:ch1/sec1/7}
    \frac{U^{n+1} -U^{n-1} }{2\tau } = F(U^{n})
\end{equation}

У такой схемы по времени есть основной недостаток: расщепления решения по нечетным и четным временным шагам \cite{ROUCH}.
Поэтому на каждом временном шаге $n$ делается фильтрация \cite{POM}:
\begin{equation} \label{eq:ch1/sec1/8}
    U^{s} = U^{n} + \frac{a}{2}(U^{n+1} - 2U^n + U^{n-1})
\end{equation}
Затем, при переходе на следующий шаг по времени, $U^{s}$ присваивается $(n-1)$-му шагу, а решение $U^{n+1}$ присваивается $n$-му.
Параметр для фильтрации выбран как $a = 0.05$ \cite{POM}.

Для донного трения в системе уравнений \cref{eq:ch1/sec1/1} используется неявная схема Эйлера по времени,
что повышает устойчивость численного алгоритма.

\section{Тестирование модели мелкой воды}

\subsection{Моделирование цунами в Японии}
С помощью вышеописанной модели мелкой воды было проведено моделирование цунами 2011 года в Японии, приведшее к катастрофе в Фукусиме, с разрешением 4 километра. Данные по начальному возвышению в очаге цунами Тохоку были любезно предоставлены М. А. Носовым \cite{NosovTsunami}.
Моделирование проводилось на 6 часов с шагом по времени 1 секунда, использовался коэффициент донного трения $n = 0.025$.
На рис. \ref{fig:Tohoku1} показано начальное возмущение и уровень в момент времени $t=30$ минут. Из рисунка видно, что, подходя к берегу, волна достигает 8-9 метров.
Также на рис \ref{fig:Tohoku2} показан уровень в моменты времени $t=1$ и $t=5$ часов. 
На рисунке видно как волна распространяется по акватории.
	
\begin{figure}[htb!]
	\begin{minipage}[h]{0.49\linewidth}
	\center{\includegraphics[scale = 0.4]{init_ssh.png}}
	\end{minipage}
	\hfill
	\begin{minipage}[h]{0.49\linewidth}
	\center{\includegraphics[scale = 0.4]{ssh_30min.png}}
	\end{minipage}
	\caption{Моделирование цунами в Японии, уровень (метры) в моменты времени $t=0$ и $t=30$ min соответственно}
	\label{fig:Tohoku1}
\end{figure}
	
\begin{figure}[htb!]
	\begin{minipage}[h]{0.49\linewidth}
	\center{\includegraphics[scale = 0.3]{ssh_1h.png}}
	\end{minipage}
	\hfill
	\begin{minipage}[h]{0.49\linewidth}
	\center{\includegraphics[scale = 0.3]{ssh_5h.png}}
	\end{minipage}
	\caption{Моделирование цунами в Японии, уровень (метры) в моменты времени $t=1$ h и $t=5$ h соответственно}
	\label{fig:Tohoku2}
\end{figure}
	
Было проведено сравнение с экспериментальными данными и сравнение результатов расчетов по нелинейным и линеаризованным уравнениям мелкой воды. 
Линейные уравнения мелкой воды представляют из себя упрощенную систему (\ref{eq:ch1/sec1/1}) в предположении $h \approx H$ при $\zeta << H$. Линеаризованные уравнения записываются следующим образом:

\begin{equation} \label{eq:1linear} 
	\begin{array}{c} 
	\displaystyle{\frac{\partial r_{x} r_{y} u}{\partial t} - r_{x} r_{y} l v + r_{y} g\frac{\partial \zeta }{\partial x} + r_x r_y \frac{g n^2}{h^{1/3}} u \sqrt{u^2 + v^2}= 0} \\ 
	
	\displaystyle{\frac{\partial r_{x} r_{y} v}{\partial t} + r_{x} r_{y} l u  +r_{x} g\frac{\partial \zeta }{\partial y} + r_x r_y \frac{g n^2}{h^{1/3}} v \sqrt{u^2 + v^2}= 0} \\ 
	
	\displaystyle{\frac{\partial \zeta}{\partial t} +\frac{1}{r_{x} r_{y} } \left(\frac{\partial u r_{y} H}{\partial x} +\frac{\partial v r_{x} H}{\partial y} \right)= 0} 
	\end{array} 
\end{equation}

\begin{table} [htbp]
\centering
%\begin{threeparttable}% выравнивание подписи по границам таблицы
\caption{Географическое расположение DART станций}\label{tab:DARTs}
	\begin{tabular}{c|c|c}
	  Name & Lon & Lat  \\
	  \hline
	  DART21418 & 148.694 & 38.711 \\ 
	  \hline
	  DART21413 & 152.117 & 30.515 \\
	  \hline
	  DART21415 & 171.847 & 50.183 \\
	  \hline
	\end{tabular}
%\end{threeparttable}
\end{table}
	%!call parallel_point_output(path2ocp, num_step, 148.694d0, 38.711d0, 'DART21418') ! DART 21418
	%!call parallel_point_output(path2ocp, num_step, 152.117d0, 30.515d0, 'DART21413') ! DART 21413
	%!call parallel_point_output(path2ocp, num_step, 152.583d0, 42.617d0, 'DART21401') ! DART 21401
	%!call parallel_point_output(path2ocp, num_step, 155.736d0, 44.455d0, 'DART21419') ! DART 21419
	%!call parallel_point_output(path2ocp, num_step, 171.847d0, 50.183d0, 'DART21415') ! DART 21415

\begin{table} [htbp]
\centering
%\begin{threeparttable}% выравнивание подписи по границам таблицы
\caption{Географическое расположение и глубина тестовых S точек}\label{tab:Ss}
	\begin{tabular}{c|c|c|c}
	  Name & Lon & Lat  & H (meters) \\
	  \hline
	  S1 & 141.017 & 38.0167 & 26 \\
	  \hline
	  S3 & 141.617 & 38.5167 & 108 \\
	  \hline
	\end{tabular}
%\end{threeparttable}
\end{table}

Сравнение проводилось в точках, где расположены DART станции (см. таблицу \ref{tab:DARTs}).
На рис. \ref{fig:DART} показано сравнение уровня, рассчитанного по линеаризованным уравнениям мелкой воды (\ref{eq:1linear}) и нелинейным (\ref{eq:ch1/sec1/1}), с данными наблюдений. Видно, что результаты моделей хорошо согласуются друг с другом, первый пик воспроизводится идентично. Также обе модели хорошо согласуются с данными реальных наблюдений. Небольшое смещение обусловлено выбранным начальным возмущением для моделей. 

Точки DART расположены довольно далеко от берега, поэтому интересно посмотреть как ведут себя модели в прибрежных акваториях. На рис. \ref{fig:S13} приведено сравнение уровня линейной и нелинейной модели для двух тестовых точек: S1, S3 (см. таблицу \ref{tab:Ss}). Из рисунка видно, что в прибрежных акваториях разница между нелинейными уравнениями мелкой воды (\ref{eq:ch1/sec1/1}) и линеаризованными (\ref{eq:1linear}) может быть уже существенной. Похожий результат был получен в работе \cite{Liu2009}.
%Видно что в этом случае вообще говоря модели ведут себя по разному.

\begin{figure}[htb!]
	\center
	\includegraphics[width=0.9\linewidth]{DART.png}
	\caption{Сравнение уровня (метры) в DART точках. Синий - нелинейные уравнения мелкой воды; красный - линеаризованные уравнения; черный - наблюдения}
	\label{fig:DART}
\end{figure}
	
\begin{figure}[htb!]
	\center
	\includegraphics[width=0.9\linewidth]{S13.png}
	\caption{Сравнение уровня (метры) в S точках. Синий - нелинейные уравнения мелкой воды; красный - линеаризованные уравнения}
	\label{fig:S13}
\end{figure}

\subsection{Моделирование экстремального шторма в Азовском море}

С помощью вышеописанной модели мелкой воды проводилось моделирование экстремального шторма на Азовском море, произошедшего 24 марта 2013 г \cite{MARESEDU}.
Пространственное разрешение использованной модели составляло 250 метров,
атмосферные данные были взяты из расчетов по WRF (Weather Research and Forecasting
Model) из работы \cite{AzovStorm}. 
Моделирование проводилось на 3 месяца с шагом по времени 1 секунда, использовался коэффициент донного трения $n = 0.025$.
Было проведено сравнение с экспериментальными данными и сравнение результатов расчетов по нелинейным (\ref{eq:ch1/sec1/1}) и линеаризованным уравнениям мелкой воды (\ref{eq:1linear}).

\begin{figure}[htb!]
	\center
	\includegraphics[width=0.85\linewidth]{EeskX.png}
	\caption{Моделирование шторма в акватории Азовского моря, сравнения уровня (метры) для поста "Ейск". 
		 Синий - нелинейные уравнения мелкой воды; красный - линеаризованные уравнения; черный - наблюдения}
	\label{fig:AS_Eesk}
\end{figure}
	
\begin{figure}[htb!]
	\center
	\includegraphics[width=0.85\linewidth]{TaganrogX.png}
	\caption{Моделирование шторма в акватории Азовского моря, сравнения уровня (метры) для поста "Таганрог". 
		 Синий - нелинейные уравнения мелкой воды; красный - линеаризованные уравнения; черный - наблюдения}
	\label{fig:AS_Taganrog}
\end{figure}

На рисунках \ref{fig:AS_Eesk}, \ref{fig:AS_Taganrog} продемонстрированы результаты расчетов (как по нелинейным уравнениям мелкой воды, так и по линеаризованным уравнениям) и данные наблюдений, полученные со станций "Ейск" и  "Таганрог" соответственно.
Из графиков видно, 
что результаты, полученные с помощью модели мелкой воды, хорошо согласуются с наблюдательными данными, и также было показано, что вклад нелинейности для этой задачи незначителен.

\FloatBarrier